\documentclass{article}
\usepackage[utf8x]{inputenc}
\usepackage{default}
\usepackage{graphicx}
\usepackage{amsmath}
\usepackage{hyperref}
\usepackage[sort&compress]{natbib}
\usepackage[a4paper,total={6in,9in}]{geometry}

\def\wp{\omega^\prime}
\def\w{\omega}
\def\wc{{\omega_{\rm C}}}
\def\ket{\rangle}
\def\bra{\langle}
\def\H{\hat{H}}
\def\P{\hat{P}_{\rm occ}}
\def\E{\varepsilon}
\def\vp{{v^\prime}}
\def\q{{\bf q}}
\def\s{\sigma}
\def\k{{\bf k}}
\def\qp{{\bf q^\prime}}
\def\G{{\bf G}}
\def\Gp{{\bf G^\prime}}
\def\rt{\tilde{r}}
\def\pt{\tilde{p}}
\def\rb{{\bf r}}
\def\rp{{\bf r^\prime}}
\def\rpp{{\bf r^{\prime\prime}}}
\def\rppp{{\bf r^{\prime\prime\prime}}}
\def\mo{$\overline{1}$}
\def\mt{$\overline{2}$}
\def\S{\mathcal{S}}
\def\Gs{\mathcal{G}}
\def\v{\mathbf{v}}
\def\symm{\left\{\mathcal{S}|\mathbf{v}\right\}}
% -------
\usepackage{soul}
\usepackage{color}
\usepackage{subfig}

\definecolor{yellow}{rgb}{1,1,0}
\definecolor{lightblue}{rgb}{0.6,0.6,0.9}
\sethlcolor{yellow}

\begin{document}
\title{Piezo-activated Trapping and Release Mechanisms of Hydrogen at Symmetric Tilt Grain Boundaries in $\alpha$-Iron}
\author{Henry Lambert}
\date{\today}
\maketitle

\section{The Elastic Dipole Tensor}
\label{sec:segregation}
Following the treatment of Freedman et. al.\cite{freedman09} we write the free energy as:
%
\begin{equation}
\label{eq:freeenergy}
f(\epsilon_{ij}, n_{d}) = f_{0} + n_{d}E_{d} + \frac{1}{2}n^{2}_{d}E_{dd} +
\frac{1}{2}\sum_{ijkl}C_{ijkl}\epsilon_{ij}\epsilon_{kl}−n_{d}\sum_{ij}\epsilon_{ij}G_{ij},
\end{equation}
%
where $E_{d}, E_{dd}, C_{ijkl}, G_{ij}$ are the defect formation energy, 
defect-defect interaction energy, elastic stiffness tensor, and the components of the elastic 
dipole tensor respectively. The derivative of the free energy 
with respect to strain then gives the stress:
%
\begin{equation}
-\frac{\partial f}{\partial \epsilon_{ij}} \equiv \sigma_{ij} = - \sum_{kl}C_{ijkl}\epsilon_{kl} + n_{d}G_{ij}.
\end{equation}
%
The derivative of the stress tensor with respect to the 
concentration at a given strain then gives the components of the
elastic dipole tensor:
%
\begin{equation}
\label{eq:elasticdipole}
\frac{\partial \sigma_{ij}}{\partial n_{d}}\Bigr|_{\epsilon} = G_{ij},
\end{equation}
%
a positive diagonal element of G indicates that the presence of
defects tends to expand the crystal along the corresponding directions.
%
A related quantity to Eq.~\ref{eq:elasticdipole} is the defect strain tensor.
This requires performing a simulation under stress control. This is achievable
by allowing the lattice vectors to relax during the course of the simulation.
The \emph{defect strain tensor} can then be defined:
%
\begin{equation}
\frac{\partial \epsilon_{ij}}{\partial n_{d}}\Bigr|_{\sigma} = \sum_{kl}S_{ijkl}G_{kl} = \Lambda_{ij},
\end{equation}
%

\subsection{Atomic Scale Stress Tensor}
Finally from Sutton Baluffi (pg. 294) we can define an atomic
scale stress tensor.
%
\begin{equation}
\delta E = \sum_{l}V^{l}\sum_{\alpha\beta}\delta\epsilon_{\alpha\beta}\sigma^{l}_{\alpha\beta}
\end{equation}
%
\begin{equation}
\sigma^{l}_{\alpha\beta} = \frac{1}{2V^{l}} \sum_{l'}f^{ll'}_{\alpha}r_{\beta}^{ll'}
\end{equation}

\begin{equation}
p^{l} = - \frac{1}{3}\sum_{\alpha} \sigma^{l}_{\alpha \alpha}
\end{equation}

\section{Hydrogenating Grains}
In order to efficiently hydrogenate grains we employ the 
Delaunay triangulation. The Delaunay triangulation is the dual transformation
to the better known Voronoi transformation. In Voronoi we have a lattice of
points and we want to partition the space around each 
point into cells which have the property
that every point in that cell is closer to the point
at the center of that cell than any other.
%
From wikipedia we have the following definition for a Delaunay
triangulation:\footnote{\url{https://en.wikipedia.org/wiki/Delaunay_triangulation}}
\begin{quote}
In a Delaunay triangulation we seek, for a set P of points in a plane, a triangulation
DT(P) such that no point in P is inside the circumcircle of any triangulation in DT(P). For
3-dimensions the Delaunay triangulation is a triangulation DT(P) such 
that no point in P is inside the circum-hypersphere of any simplex in DT(P).
\end{quote}
In a regular iron lattice octahedral position comes at the circumcenter of
the circles represented in Fig.~\ref{fig:delaunay}. 
%
\begin{figure}[!tbp]
\begin{center}
\includegraphics[scale=0.8]{FeHDelaunay.png}
\caption{Octahedral position viewed as the circumcenter of the Delaunay triangulation 
of an iron lattice. Octahedral position in red, tetrahedral positions 
blue triangles, broken lines indicate relevant circumsphere.
\label{fig:delaunay}}
\end{center}
\end{figure}
%
We can then use octahedral sites identified in this way as a lattice, 
and by defining a suitable basis, we can obtain all the
tetrahedral positions in a volume slice. The Delaunay 
triangulation at a grain boundary is also useful. For a generic
grain boundary there is no longer a clear distinction between 
octahedral and tetrahedral sites, but rather interstitial sites
of varying volume. By performing a Delaunay triangulation we 
are able to systematically identify tetrahedral
positions in the regular lattice and rank interstitial sites at 
the grain boundary by volume. Tesselations
of the lattice and grain boundary with varying planar concentrations 
of hydrogen allow us to efficiently compute the grain boundary 
segregation energy and any possible inter-hydrogen co-operative 
effects during hydrogen platelet formation at an interface.


\section{Elastic Fields of Interfaces}

\subsection{Frank's Formula}
The dislocation spacing can be estimated using Frank's formula.
%
\begin{equation}
d = \frac{|\mathbf{b}|}{2 \sin(\Delta\theta/2)}
\end{equation}
%
\subsection{Isolated Edge Dislocations}
The stress field $\sigma$ in the environment of an edge dislocation.
\begin{equation}
\label{eq:edgexx}
\sigma_{xx} = - \frac{\mu b_{x}}{2\pi(1-\nu)}\frac{y(3x^{2}+y^{2})}{(x^{2}+y^{2})^{2}}
\end{equation}

\begin{equation}
\label{eq:edgeyy}
\sigma_{yy} = \frac{\mu b_{x}}{2\pi(1-\nu)}\frac{y(x^{2}-y^{2})}{(x^{2}+y^{2})^{2}}
\end{equation}

\begin{equation}
\label{eq:edgexy}
\sigma_{xy} = \frac{\mu b_{x}}{2\pi(1-\nu)}\frac{x(x^{2}-y^{2})}{(x^{2}+y^{2})^{2}}
\end{equation}

\begin{equation}
\label{eq:edgezz}
\sigma_{zz}=\nu(\tau_{xx}+\tau_{yy})=\frac{-\mu b_{x}\nu}{2\pi(1-\nu)}\frac{y}{(x^{2}+y^{2})}
\end{equation}


\begin{equation}
\label{eq:edgexz}
\sigma_{xz} = \sigma_{yz} = 0
\end{equation}

Where,for iron at room temperature, typical values for the 
shear modulus are $\mu=52.5$~Pa and for the Poisson ratio $\nu=0.22-0.31$.\cite{rayne61}

\subsection{Network of Edge Dislocations}
For a network of edge dislocations the sum can be performed:
%
\begin{equation}
\label{eq:edgexx}
\sigma_{xx} = - \frac{\mu b_{x}}{2\pi(1-\nu)}\sum_{n=-\infty}^{\infty}\frac{(Y-n)(3X^{2}+(Y-n)^{2})}{(X^{2}+(Y-n)^{2})^{2}},
\end{equation}
%
where $X=x/D$, $Y=y/D$.

\begin{equation}
\label{eq:edgexx}
\sigma_{yy} = \frac{\mu b_{x}}{2\pi(1-\nu)}\sum_{n=-\infty}^{\infty}\frac{(Y-n)(X^{2}-(Y-n)^{2})}{(X^{2}+(Y-n)^{2})^{2}},
\end{equation}
%
\begin{equation}
\label{eq:edgexx}
\sigma_{xy} = \frac{\mu b_{x}}{2\pi(1-\nu)}\sum_{n=-\infty}^{\infty}\frac{X(X^{2}-(Y-n)^{2})}{(X^{2}+(Y-n)^{2})^{2}},
\end{equation}
%
If the Burgers vector is oriented purely along $[b_{x},0,0]$ 
the stress tensor has the following components:\footnote{\cite{sutton95}}
%
\begin{align*}
\sigma_{0}  &= \frac{\mu b_{x}}{2D(1-\nu)(\cosh(2\pi X)-\cos(2\pi Y))^{2}},\\
\sigma_{xx} &= -\sigma_{0}\sin(2\pi Y)(\cosh(2\pi X) - \cos(2\pi Y)+2\pi X\sinh(2\pi X)),\\
\sigma_{xy} &= \sigma_{0}2\pi X(\cosh(2\pi X)\cos(2\pi Y)-1),\\
\sigma_{yy} &= -\sigma_{0}\sin(2\pi Y)(\cosh(2\pi X) - \cos(2\pi Y) + 2\pi X\sinh(2 \pi X)).
\end{align*}
%
In the case the Burgers vector is oriented along $[0,b_{y},0]$ the dislocation network
stress tensor has components:
\begin{align*}
\sigma_{xx} &= \sigma_{0}\sin(2\pi Y)(\cosh(2\pi X) - \cos(2\pi Y)+2\pi X\sinh(2\pi X)),\\
\sigma_{xy} &= -\sigma_{0}2\pi X(\cosh(2\pi X)\cos(2\pi Y)-1),\\
\sigma_{yy} &= \sigma_{0}\left[2\sinh(2\pi X)(\cosh(2\pi X)- \cos(2\pi Y)) - 2 \pi X(\cosh(2\pi X)\cos(2\pi Y)-1)\right]
\end{align*}

\subsection{Grain Boundary Algebra}
Just for reference, the algebra is similar for all other components of the stress, and
can be extended to cases where we have a traveling edge dislocation i.e. instead of
a sum over n $n$ we have a factor like (y-vt)....
First an auxiliary function is introduced:
%
\begin{equation}
f(z) =  \pi\cot(\pi z)\frac{(Y-z)(3x^{2} + (Y-z)^{2})}{(X^{2} + (Y-z)^{2})^{2}}
\end{equation}
%
This has simple poles at $z = 0, \pm 1, \pm 2,...$. It also has two 2nd 
order poles at  $z_{1}=iX+Y$, $z_{2}=-iX+Y$. The residue at $z_{2}$ can
be found from the following expression: 
%
\begin{equation}
{\rm Res}(z_{2}) = \lim_{z\rightarrow z_{2}}\frac{d}{dz}(z-z_{2})^{2}
\frac{\cot(\pi z)(Y-z)(3X^{2}+(Y-z)^2)}{(z-z_{1})^{2}(z-z_{2})^{2}}
\end{equation}


\begin{align}
\label{eq:2ndstep}
{\rm Res}(z_{2})&=\frac{\pi^{2}}{\sin^{2}(\pi z_{2})}\left[\frac{(Y-z_{2})(3x^{2}+(Y-z_{2})^{2})}{(z_{2}-z_{1})^{2}}\right]\\
  &+\frac{\pi\cot(\pi z_{2})}{(z_{2}-z_{1})^{2}}\left[-(3x^{2} + (Y-z_{2})^{2})-2(Y-z_{2})^{2}\right] \\
  &+\frac{2\pi\cot(\pi z)(Y-z_{2})(3X^{2}+(Y-z_{2})^{2})}{(z_{2} - z_{1})^{3}}
\end{align}
%
The second term in Eq.~\ref{eq:2ndstep} evaluates to zero and the remaining terms can be simplified to:
%
\begin{equation}
{\rm Res}(z_{2}) = \frac{\pi}{4\sin^{2}(\pi(Y-iX))}(i\pi X - \sin(2\pi(Y-iX)))
\end{equation}
%
The residue at $z_{1}$ can be found in a similar manner.

If we construct a set of contours as depicted in Fig.~\ref{fig:polediagram}.
\begin{figure}[!tbp]
\begin{center}
\includegraphics[scale=1.0]{stresspolediagram.pdf}
\caption{Contours to determine sum. The lines of poles from the $\cot(\pi z)$ 
factor can be equated to the difference of the two 
second order residues.\label{fig:polediagram}}
\end{center}
\end{figure}

\begin{equation}

\end{equation}

We can write down the sum as the difference of the two second order residues.

\subsection{Strain Components}
We can then use Hooke's law for an isotropic medium to translate the
components of the stress tensor into strain components:
%
\begin{align*}
\sigma_{xx} &= 2\mu \epsilon_{xx}+\lambda e,\\
\sigma_{yy} &= 2\mu \epsilon_{yy}+\lambda e,\\
\sigma_{zz} &= 2\mu \epsilon_{zz}+\lambda e,\\
\sigma_{xy} &= 2\mu \epsilon_{xy},\\
\sigma_{yz} &= 2\mu \epsilon_{yz},\\
\sigma_{zx} &= 2\mu \epsilon_{zx}
\end{align*}
%
where $e=\epsilon_{xx}+\epsilon_{yy}+\epsilon_{zz}$ and:
%
\begin{equation}
\lambda = \frac{2\nu\mu}{(1-2\nu)}
\end{equation}
%

\subsection{H Concentration}
To determine which symmetric tilt boundaries might be of particular importance
for hydrogen embrittlement we consider the elastic field of a grain boundary
interface.

We focus on the contribution of the coupling between hydrogen and the strain field 
to the free energy in Eq.~\ref{eq:freeenergy}. Given an estimate of the elastic
dipole tensor for hydrogen and expressions for the strain field in the vicinity of a grain boundary
we can calculate the variation in the local concentration of the hydrogen atmosphere. This model
is effectively searching for a strain induced Cottrell atmosphere in the vicinity of 
a grain boundary \cite{cottrell49, louat56}.

In order to obtain reasonable approximations to the strain  
field we restrict the analysis to symmetric tilt boundaries
with geometries given in Fig.~\ref{fig:symmtilt}.
%
\begin{figure}[!tbp]
\begin{center}
\includegraphics[scale=1.0]{HGBIntegrationRegion.eps}
\caption{Integration regions for the dislocation network at a symmetric tilt boundary for two different
orientations of the burgers vector. a$)$ Edge dislocations with a burger's vector oriented along 
$(b_x,0,0)$, and b$)$ Edge dislocation with a burger's vector oriented along $(0,b_y,0)$.\label{fig:symmtilt}}
\end{center}
\end{figure}
%
It is well established that hydrogen will migrate to regions of dilational strain,
see Ref.~\cite{ramasubramaniam09} and references therein. The energetic preference
for hydrogen migrating to regions of dilational strain can then be described:\cite{}
%
\begin{equation}
\label{eq:hdistrib}
[H] = \frac{1}{A}\int_{-D/2}^{D/2}\int_{\rm r_{1}}^{\rm r_{2}}C_{0}{\rm Exp}[{\frac{-\sum_{\rm ij}\G_{\rm ij}\epsilon_{\rm ij}(x,y;D)}{{\rm k_{B}T}}}]dxdy.
\end{equation}
%
Where D is the dislocation spacing, $G_{\rm ij}$ is the elastic dipole tensor, $C_{0}$
is the bulk concentration of hydrogen in the material, and $\epsilon_{ij}(D)$ 
is the elastic strain tensor with its dependence on 
the dislocation spacing made explicit, the quantity $A$ normalizes the region. 
%
\subsection{Twist Boundaries}
Where the Burgers vector is oriented along the z axis we must consider screw dislocations
at the interface. The stress fields associated with equally spaced screw dislocations
takes the form:
%
\begin{align*}
\sigma_{xz} &= -\frac{\mu b_{z}}{2D} \frac{\sin(2\pi Y)}{\cosh(2\pi X) - \cos(2\pi Y)}\\
\sigma_{yz} &=  \frac{\mu b_{z}}{2D} \frac{\sinh(2\pi X)}{\cosh(2\pi X) - \cos(2\pi Y)}
\end{align*}

The previous formulas are derived from the stress fields of isolated 
screw dislocations:
%
\begin{align*}
\sigma_{xz} = -\frac{\mu b_{z}}{2\pi}\frac{y}{x^{2} + y^{2}}\\
\sigma_{yz} =  \frac{\mu b_{z}}{2\pi}\frac{x}{x^{2} + y^{2}}\\
\sigma_{xy} =  \sigma_{xx}=\sigma_{yy}=\sigma_{zz}=0
\end{align*}
%
The displacement field for the isolated screw dislocation is:
%
\begin{equation}
u_{z} = -\frac{b_{z}}{2\pi} \tan^{-1}(x/y),
\end{equation}
%
there is no in plane displacement field. In the absence of off diagonal 
elements of the hydrogen elastic dipole tensor there is no coupling 
to a pure screw dislocation. In this case the only relevant effect 
will stem from the H-dislocation core interaction.

\subsection{Concentration Variation}
Eq.~\ref{eq:hdistrib} is meant to approximate the deviation of hydrogen concentration
from its bulk value in the immediate vicinity of a boundary.
In order to evaluate the magnitude of the integral appearing in Eq.~\ref{eq:hdistrib}
for different dislocation spacings we need to introduce some reasonable 
parameters and assumptions. 

We choose representative values of all the parameters entering the linear elastic model,
i.e. $|{\rm b}|=2.45$\AA, and $\nu=0.28$. We estimate the most important contribution
to the elastic dipole energy will come from the diagonal elements. Hence the sum in the
exponent of Eq.~\ref{eq:hdistrib} needs only to be traced over. 
This further simplifies evaluation of the integration since
the trace of the matrix is invariant under rotation. We therefore 
can estimate the concentration over the entire misorientation angle range. 
We estimate the magnitude of the diagonal elements of the elastic dipole tensor using the EAM
where we find a diagonal component of $G_{ii}=4.45$~eV.\footnote{\cite{ramasubramaniam09}}

The result of the integration is insensitive in its shape and magnitude to reasonable variations 
of these parameters. To perform the integration numerically we must choose cutoffs
for the integration range along $x$. We set the bottom of the integration region at 
the dislocation core radius, $4.5$~\AA, and extend the integration $60$~\AA into the the bulk. 
The upper cutoff is chosen as representative of 2 times the largest dislocation 
spacing encountered in the atomistic simulations.
At this distance the stress field will have decayed to less than 1\% 
of its peak value for the largest dislocation
spacing considered.\footnote{\cite{sutton95}}

Using these parameters and approximations the integration can be 
handled numerically to produce the results in Fig.~\ref{fig:intresult}.
%
\begin{figure}[!tbp]
\begin{center}
\begin{minipage}[t]{7.5cm}
\input{symmtiltHbx.tex}
\end{minipage}
\begin{minipage}[t]{7.5cm}
\input{symmtiltHby.tex}
\end{minipage}
\caption{Variation in the hydrogen concentration in the vicinity of two
symmetric tilt grain boundary dislocation networks. The left panel is for
a dislocation network with the Burgers vector oriented purely along x (panel (a) in Fig.~\ref{fig:symmtilt})
and for the for Burgers vector oriented along y (panel (b) in Fig.~\ref{fig:symmtilt}). 
In the first case there is no long range strain
away from the boundary and the hydrogen concentration increases quasi-linearly with dislocation
spacing before leveling off at 30~\AA and beginning to decay albeit slowly. The situation for the second
dislocation network is slightly different. In this case the boundary must be composed 
of two distinct dislocations with opposite components of the burger's vector along y
in order to create a boundary with no long range strain. The variation in the 
local hydrogen concentration is a much more sharply peaked function of dislocation spacing
and motivates a search for boundaries with a geometry of this type. \label{fig:intresult}}
\end{center}
\end{figure}

We can observe a few features in Fig.~\ref{fig:intresult}. For the case
of the $b_{x}$ orientation of the Burgers vector the hydrogen concentration
increases steadily with an increase in dislocation spacing. Boundaries
of this type have symmetric stress fields either side of the boundary plane 
which decay into the bulk. The 'capacity' of the grain boundary to store hydrogen increases with
the available actively stressed region (shaded blue region in Fig.~\ref{fig:symmtilt}). 
The decay of the stress field parallel to the boundary is determined by the 
dislocation spacing. Beyond 30~\AA the dislocations are sufficiently spaced such 
that the tensile field has decayed and the concentration excess saturates.

For the alternative orientation of the Burgers vector the results are somewhat different.
In this case there is a compressive and tensile region on either side of the grain boundary
interface. We are interested in the accumulation of hydrogen so we restrict the integration
to the tensile half plane to obtain the results in Fig.\ref{fig:intresult} (b). Here there
is a sharp initial peak in the hydrogen accumulation which decays rapidly with dislocation spacing.

The magnitude of this effect is most strongly depedent on the magnitude of
the components of the elastic dipole tensor. Hence the relevant temperature
ranges where the effect may by most keenly observed will have a strong material
dependence.

We note a more detailed solution of Eq.~\ref{eq:hdistrib} that includes off diagonal elements of
the elastic dipole tensor might give a more complete picture of the relative accumulation and
angular distribution of H atoms in the immediate vicinity of a grain boundary: in particular at low 
temperatures. However for the present purposes Eq.~\ref{eq:hdistrib}, allows us to sharpen
our search criterion for grain boundaries of particular relevance in hydrogen embrittlement.
In particular we are on the look out for boundaries where there is a transition in the 
dislocation spacing of the interface for the distances where Eq.~\ref{eq:hdistrib} varies most rapidly.
This could quickly relieve the stress field and the grain boundary would act
as a source of hydrogen. Alternatively boundaries which, when 
strained, might form a new minimum energy structure with a denser dislocation
spacing could act as a hydrogen sink. The search for
grain boundaries that demonstrate this behaviour is detailed in the next section.

The present analysis neglects the contribution of the core region of the boundary, or
the individual dislocations, to hydrogen trapping. It also 
neglects any effects related to `self-consistency'; i.e. 
co-operative feedback effects of hydrogen accumulation in the region
of a grain boundary. In order to include these effects atomistic simulations 
are now required to estimate the energy depth of the grain boundary core trapping sites.
We will also be interested in the evolution of the core of the grain boundary dislocation
structures under strain and in the presence of hydrogen. 

\section{Grain Boundary Database}
The minimum energy structures for the grain boundary database are
generated according to the procedure of Rittner and Seidman \cite{rittner96}. We 
typically investigate 300-500 structures for each grain boundary.
The generation of a large set of grain boundaries has been performed recently in
Ref.~\cite{tschopp12} in a study on vacancy and interstitial absorption in iron grain
boundaries, and Ref.~\cite{wang13} where the hydrogen embrittlement of grain boundaries was
studied. One of the results of this paper is that the duplication of this effort should be unnecessary. 
To this end we have made the relaxed grain boundary structures for the entire misorientation angle
range of the [001], [110], and [111] symmetric tilt grain boundaries available online. The database
is indexed according to the orientation axis, misorientation angle, boundary plane,
microscopic rigib body translations, and atomic cutoff criterion. The database is further divided according 
to the interatomic potential employed i.e. EAM, tightbinding, DFT etc. 
Using this indexing system the database can easily be extended 
to any arbitrary grain boundary system. The framework for the archiving 
and accessibility of this grain boundary database is general and extensible so 
research groups with similar databases, for any material, could easily 
upload their grain boundary datasets into the framework.

\subsection{Grain Boundary Segregation Energy}
The linear elastic solutions are no longer valid at the core of a dislocation.
Atomistic simulations of interactions between Hydrogen and isolated screw and
edge dislocation have been extensively studied 
at a variety of levels of theory \cite{taketomi08, kimizuka11}.

In order to treat the energetics and dynamics of hydrogen in this region 
we require atomic scale modelling. Initially we treat the 
Fe-H interactions using the modified EAM potential of Ramasumabramaniam et. al.
The quantity we are interested in calculation is the segregation energy. 
This is the difference in energy between a solute atom 
being in the bulk and at a grain boundary site:
%
\begin{equation}
\Delta e_{{\rm seg}} = \Delta E^{\rm d}_{{\rm gb}} - \Delta E^{\rm d}_{{\rm bulk}}
\end{equation}
%
By iterating over the relaxed grain boundary structures we can examing the energetics
of the hydrogen at the boundary interface and in the bulk. This screening
process should allow us to determine which, if any, boundary acts as a particularly 
favourable sink for hydrogen segregation.

\subsection{Oriani Traps}
Oriani provides a means for calculating the diffusivity of hydrogen in a lattice
containing traps where there is an equilibrium between normal lattice sites and
trap sites. The trap depth necessary to obtain this equilibrium is also calculated.   

The equilibrium for traps is given by:
\begin{equation}
%\H \leftrightharpoons H_{x} + E_{x} %\qquad E_{x} < 0
\end{equation}

With the equilibrium constant for the above reaction given by:
%
\begin{equation}
\label{eq:equilibrium}
K = \frac{1}{\Theta_{L}}\frac{\Theta_{x}}{1-\Theta_{x}}
\end{equation}

For an infinite slab of half width ($l$), in a time period
$10^{-2}\frac{l^{2}}{D_{L}}$, the mean concentration will be:
%
\begin{equation}
\frac{C_{m} - C_{L,0}}{C_{L,0}}=-0.110.
\end{equation}
%
In a small slab located at x, where $x/l=0.95$ ($C_{L}-C_{L,0}/C_{L,0}=-0.7)$.
For equilibrium to be maintained the change in trap concentration must equal:
%
\begin{equation}
\delta c_{x} = \frac{N_{x}}{N_{L}}(1-\Theta_{x})^{2}Kc_{L}
\end{equation}
%
For a slice of the slab near the surface this becomes:
%
\begin{equation}
\delta c_{x} = \frac{N_{x}}{N_{L}}(1-\Theta_{x})^{2}K(0.7c_{0,L})
\end{equation}
%
Where $\delta c_{x}$ is number of trapped atoms per unit volume
that must become untrapped in a time equal to 
or less than $10^{-2}l^{2}/D_{L}$ for local equilibrium to be maintained.

If $\mathcal{R}$ is the rate of escape from a trap site local equilibrium is met by:
%
\begin{equation}
\label{eq:timeconst}
\frac{\delta c_{x}}{\mathcal{R}N_{x}\Theta_{x}} \le 10^{-2}l^{2}/D_{L}
\end{equation}
%
The escape rate can be modeled:
%
\begin{equation}
\mathcal{R} = \nu {\rm exp}[(\Delta E_{x} - E_{a} - E')/RT]
\end{equation}
%
where $\nu$ is the vibration frequency of the trapped hydrogen.

\begin{equation}
D_{L} = \lambda^{2}\nu {\rm exp}(-E_{a}/RT)
\end{equation}
and
\begin{equation}
K = {\rm exp}(-\Delta E_{x}/RT)
\end{equation}

The trap escape rate can then be modeled using the expressions for the
equilibrium constant and the diffusibility as:
\begin{equation}
\mathcal{R} = K^{-1}\lambda^{-2}D_{L}e^{-E'/RT}.
\end{equation}
The final condition for local equilibrium becomes:
%
\begin{equation}
\label{eq:localequil}
70K(\lambda/l)^{2}e^{E'/RT}(1-\Theta_{x}) \le 1
\end{equation}
%
Plugging in reasonable values for steel $\Theta_{x}=10^{-1}$,
$\lambda \approx 2\times 10^{-10}$m and $l=10^{-3}$m and restricting the 
model to $E'=0$ one finds that solving Eq.~\ref{eq:localequil} gives
values of: $K\le 4 \times 10^{11}$ which at $T=300K$ means the trap depth
would have to be $-\Delta E_{x} \le 16.0$ kcal/mol. 1 kcal=4184 J or 
in joules $E_{x} \le 66.9 kJ/mol$.

\subsection{Screening Database for Traps}

\section{Dynamical Simulations}
Of particular relevance is the work of Taketomi et. al. in Ref.\cite{taketomi11} on the competition
between edge dislocation motion and hydrogen diffusion. 

\section{Practical Calculations}
We can now follow this procedure: 
\begin{itemize}
\item Compute the Macroscopic and atomic stress tensor at a 
given hydrogen concentration using the infinitesimal strain.
\item Repeat for a range of hydrogentandard approach of Rittner et. al. is followed. 
Other workers in the field have shown that the typical parameter 
space that one is required to search is on the order of %(rbt*rcut_steps)=4*8*85.
concentrations.
\item Plot $\sigma_{ij}$ vs. $[H]$ and obtain $G_{ij}$ from the slope of this line.
\end{itemize}

\section{Fracture Simulations}
\subsection{Low Index Fracture Surfaces}
We have two references for our fracture simulations. The first is \cite{moller14}.
Our purpose for doing these DFT calculations is two fold. Firstly we 
wish to generate a dataset of Fe-H forces that includes the electronic degrees
of freedom.

The second is the Curtin and Song model.
We summarize Bitzek's results here:
%
\begin{table}[!tb]
\begin{tabular}{c c c c c}
\hline
Crack Geometry      & Event     & Plane                    & K$_{G}$ & K$_{\rm emit}$ \\
$(100)[001]$        & B         & $(100)/(1\bar{1}0)^{a}$  &  0.92   &  1.31 \\
$(100)[011]$        & B         & $(100)$                  &  0.95   &  1.62 \\
$(110)[001]$        & B         & $(110)$                  &  0.84   &  1.31 \\
$(110)[1\bar{1}0]$  & B         & $(110)$                  &  0.92   &  1.15 \\
$(111)[1\bar{1}0]$  & Twinning  & $(11\bar{2})$            &  0.99   &  1.19 \\
$(111)[11\bar{2}]$  & D         & $(\bar{1}10)$            &  0.97   &  1.09 \\   
\hline
\hline
\end{tabular}
\caption{B: Brittle fracture, D dislocation emission, (a) crack kinks on to (110) plane.
For the first four cases we have uniform brittle fracture
so it's possibly not too interesting to investigate them further for embrittling
effects. The remaining two (111) fracture geometries have some interesting properties
and we will be interested in those. For $(111)[1\bar{1}0]$ crack system deformation 
twins are emitted independent of the potential used.
There is then subsequent glide of $\frac{a_{0}}{6}[111](11\bar{2})$ twinning
dislocations on adjacent twin planes. For the $(111)[11\bar{2}]$ crack system
$\frac{a_{0}}{2}[111](\bar{1}10)$ edge dislocations are emitted.}
\end{table}

Note Bitzek says something complicated about $(111)[1\bar{1}2]$ along the lines of
$[1\bar{1}2]$ is not a mirror plane the so the validity of a
linear elastic solution is doubtful due to a lack of orthotropic elasticity. So
they don't treat it further. Curtin on the other hand doesn't seem to worry about this
too much. I'm inclined not to worry about it either. The only effect is we won't have
an accurate estimate of K, the embrittlement de-embrittlement phenomena will be unaffected.
Speaking of K they define this as  $K_{G}=\sqrt{G_{G}B^{-1}}$ and perform incremental
loading as $\delta K_{I}=0.007 MPa \sqrt{m}$. Need to ask James about equivalencies in quippy.
Also there Gordon et. al. found $(110)[1\bar{1}0]$ to be ductile with rectangular configuration
and $25x25x2{\rm nm}^{3}$ geometry whereas Bitzek finds it to be brittle. So need to be careful we
justify the geometry. Probably best to setup a standard load, then outer product with
strain rates and initial thermalized hydrogen concentrations in the vicinity and temperatures.
That will settle us for standard low index miller index plane fracture.

\subsection{Grain Boundaries}
This is trickier and time/calculation dependent.

\section{Fracture Simulations}
\begin{figure}[!tbp]
\begin{center}
\includegraphics[width=\textwidth]{111frac.png}
\caption{Fracture geometries. In panel (a) we expect to see deformation twins
emitted, with glide along the $\frac{a_{0}}{6}[111](11\bar{2})$ system. In panel
(b) we anticipate the emission of $\frac{a_{0}}{2}[111](\bar{1}10)$ edge dislocations.
\label{fig:fracgeoms}}
\end{center}
\end{figure}

\scriptsize
\bibliographystyle{unsrtnat}
\bibliography{refs}
\end{document}
